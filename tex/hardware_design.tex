\section{Hardware Design: Efficient Architecture}\label{sec:hardware}

In this section, we investigate the hardware level techniques used in state-of-the-art FPGA based neural network accelerator design to achieve high performance and high energy efficiency. We classify the techniques into 3 levels: computation unit level, loop mapping level, and memory system level.

\subsection{Computation Unit Designs}

Computation unit level design affect the peak performance of the neural network accelerator. With a certain FPGA chip, the available resource is limited. A smaller computation unit design means more computation units and higher peak performance. A carefully designed computation unit array can also increase the working frequency of the system and thus improve peak performance.

\subsubsection{Low Bit-width Unit}
Reduce the number of bit-width for computation is a direct way to reduce the size of computation units. The feasibility of using less bits comes from the quantization methods as introduced in section~\ref{sec:software:quant}. Most of the state-of-the-art FPGA designs replace the 32-bit floating point units with fixed point units. Podili, et al.~\cite{podili2017fast} implements 32-bit fixed point untis for the proposed system. 16-bit fixed point units are widely adopted in \cite{qiu2016going, li2016high, xiao2017exploring, guan2017fp, zhang2016caffeine}. ESE~\cite{han2017ese} adopts 12-bit fixed-point weight and 16-bit fixed point neurons design. Guo, et al.~\cite{guo2017angel} use 8-bit units for their design on embedded FPGA. Recent work is also focusing on extremely narrow bit-width design. Experiments in \cite{nurvitadhi2016accelerating} shows that FPGA implementation of Binarized Neural Network (BNN) outperforms that on CPU and GPU. 

\subsubsection{Fast Convolution Unit}

\subsubsection{Frequency Optimization Methods}


\subsection{Loop Mapping Strategies}

\subsection{Memory System}