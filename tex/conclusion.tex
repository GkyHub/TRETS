\section{Conclusion}\label{sec:conclusion}

In this paper, we review state-of-the-art neural network accelerator designs and summarize the techniques used. According to the evaluation result in section~\ref{sec:evaluation}, with software hardware co-design, FPGA is able to achieve $13\times$ better energy efficiency than state-of-the-art GPU while using $30\%$ power with conservative estimation. This shows that FPGA is a promising candidate for neural network acceleration. We also review the methods used for flexible accelerator design, which shows that current development flow is able to achieve both high performance and run-time network switch.

But there is still gap between current designs and the estimation. Combining all the techniques requires software-hardware co-design. Using quantization and weight reduction together while maintaining the performance is challenging. Scaling up the design is another problem. Future work should focus on solving these challenges. There is still $10\times$ potential performance gain from using $1/2$ bits for neuron and weight representation. Future work should try to improve the model accuracy in these cases.